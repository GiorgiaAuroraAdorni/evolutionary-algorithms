\section{Covariance Matrix Adaptation Evolution Strategy \\(CMA-ES)}

The last algorithm implemented is the \textbf{Covariance Matrix Adaptation Evolution Strategy (CMA-ES)}. 

Different execution of the algorithm has been performed using the same parameters employed with the first method. In Table \ref{tab:cmaes-param} are shown the different combinations of parameter used.

\begin{figure}[htb]
	\centering
	
	\begin{tabular}{lccc}
		\toprule
		\textbf{experiment} & \textbf{population} & \textbf{elite} &
		\textbf{generations} \\
		\midrule
		\texttt{baseline 						}	 & 100 	& 0.20 	& 100\\
		\texttt{pop\_size-1000 					}	 & 1000 	& 0.20 	& 100\\
		\texttt{pop\_size-3000 					}	 & 3000 	& 0.20 	& 100\\
		\texttt{elite-40 						}	 & 100 	& 0.40 	& 100\\
		\texttt{elite-30 						}	 & 100 	& 0.30 	& 100\\
		\texttt{elite-10 						}	 & 100 	& 0.10 	& 100\\
		\texttt{pop\_size-1000+elite-40 		}	 & 1000 	& 0.40 	& 100\\
		\texttt{pop\_size-1000+elite-30 		}	 & 1000 	& 0.30 	& 100\\
		\texttt{pop\_size-1000+elite-10 		}	 & 1000 	& 0.10 	& 100\\
		\texttt{pop\_size-3000+elite-40 		}	 & 3000 	& 0.40 	& 100\\
		\texttt{pop\_size-3000+elite-30 		}	 & 3000 	& 0.30 	& 100\\
		\texttt{pop\_size-3000+elite-10 		}	 & 3000 	& 0.10 	& 100\\
		\texttt{iter-200 						}	 & 100 	& 0.20 	& 200\\
		\texttt{iter-50 						}	 & 100 	& 0.20 	& 50\\
		\texttt{iter-30 						}	 & 100 	& 0.20 	& 30\\
		\texttt{iter-200+elite-30 				}	 & 100 	& 0.30 	& 200\\
		\texttt{iter-200+pop\_size-3000+elite-30} 	 & 3000 	& 0.30 	& 200\\	
		\bottomrule
	\end{tabular}
	\captionof{table}{CMA-ES parameters}
	\label{tab:cmaes-param}
\end{figure}

The algorithm has been run 3 times for both of test functions. 
In order to evaluate the performance, for each pair of experiment and test function, the best and the worse fitness for each generation (averaged over 3 runs) has been plotted. 

In Tables \ref{tab:cmaes-performance-s} and \ref{tab:cmaes-performance-r} are summarised the results. 

\begin{figure}[htb]
	\centering
	
	\begin{tabular}{lccc}
		\toprule
		\textbf{experiment} & \textbf{best fitness} & \textbf{worse fitness} & \textbf{avg run time} \\
		\midrule
		\texttt{baseline 						}		&	567.81 &	 567.82 &	  0.63 sec \\
		\texttt{pop\_size-1000 					}		  &	 92.09 &	  92.11 &	  1.14 sec \\
		\texttt{pop\_size-3000 					}		  &	  0.01 &	   0.01 &	  2.56 sec \\
		\texttt{elite-40 						}		&	586.48 &	 586.48 &	  0.75 sec \\
		\texttt{elite-30 						}		&	570.19 &	 570.21 &	  0.86 sec \\
		\texttt{elite-10 						}		&	657.87 &	 657.87 &	   0.6 sec \\
		\texttt{pop\_size-1000+elite-40 		}	 &	 56.86 &	  57.09 &	   1.2 sec \\
		\texttt{pop\_size-1000+elite-30 		}	 &	  40.8 &	  40.84 &	   1.2 sec \\
		\texttt{pop\_size-1000+elite-10 		}	 &	132.82 &	 132.83 &	  1.09 sec \\
		\texttt{pop\_size-3000+elite-40 		}	 &	   0.0 &	   0.01 &	  2.58 sec \\
		\texttt{pop\_size-3000+elite-30 		}	 &	   0.0 &	    0.0 &	  2.43 sec \\
		\texttt{pop\_size-3000+elite-10 		}	 &	   6.4 &	   6.41 &	   2.3 sec \\
		\texttt{iter-200 						}		&	586.27 &	 586.27 &	  0.83 sec \\
		\texttt{iter-50 						}		 &	609.34 &	 609.35 &	  0.47 sec \\
		\texttt{iter-30 						}		 &	602.65 &	 602.69 &	  0.47 sec \\
		\texttt{iter-200+elite-30 				}	&	 554.4 &	  554.4 &	  0.84 sec \\
		\texttt{iter-200+pop\_size-3000+elite-30} 	 &	   0.0 &	    0.0 &	  4.87 sec \\
		
		\bottomrule
	\end{tabular}
	\captionof{table}{Sphere CMA-ES performance}
	\label{tab:cmaes-performance-s}
\end{figure}

For the sphere function, it is possible to see that simply using a large enough population size the algorithm converges.

In Figure \ref{fig:cmaes-s-fitness/baseline} is plotted the best and the worse fitness for each generation (averaged over 3 runs) for the baseline model. Below, in Figure \ref{fig:cmaes-s-fitness/3000} is plotted the fitness of the model that performs better with the sphere function.

 
\begin{figure}[H]
	\centering
	\begin{minipage}[b]{.6\textwidth}
		\includegraphics[width=\linewidth]{../src/out/cma_es/baseline/sphere/fitness.pdf}	
	\end{minipage}
	\caption{Sphere fitness \texttt{baseline}}
	\label{fig:cmaes-s-fitness/baseline}
\end{figure}
\begin{figure}[H]
	\centering
	\begin{minipage}[b]{.6\textwidth}
		\includegraphics[width=\linewidth]{../src/out/cma_es/pop_size-3000/sphere/fitness.pdf}	
	\end{minipage}
	\caption{Sphere fitness \texttt{pop\_size-3000}}
	\label{fig:cmaes-s-fitness/3000}
\end{figure}

For the rastrigin function, it is necessary not only to increase the size of the population but also the number of generations to allow the algorithm to converge.

\begin{figure}[htb]
	\centering
	
	\begin{tabular}{lccc}
		\toprule
		\textbf{experiment} & \textbf{best fitness} & \textbf{worse fitness} & \textbf{avg run time} \\
		\midrule
		\texttt{baseline 						}		&              1394.23 &	1394.23 &	  0.95 sec \\
		\texttt{pop\_size-1000 					}		&	446.51 &	 447.11 &	  1.53 sec \\
		\texttt{pop\_size-3000 					}		&	120.04 &	 126.67 &	  3.51 sec \\
		\texttt{elite-40 						}		&              1298.12 &	1298.27 &	  0.86 sec \\
		\texttt{elite-30 						}		&              1433.35 &	1433.38 &	  0.87 sec \\
		\texttt{elite-10 						}		&              1385.37 &	1385.37 &	  0.82 sec \\
		\texttt{pop\_size-1000+elite-40 		}	 &	443.78 &	 453.89 &	   1.6 sec \\
		\texttt{pop\_size-1000+elite-30 		}	 &	390.73 &	 393.75 &	  1.58 sec \\
		\texttt{pop\_size-1000+elite-10 		}	 &	550.56 &	 550.63 &	  1.51 sec \\
		\texttt{pop\_size-3000+elite-40 		}	 &	400.15 &	 751.57 &	  3.42 sec \\
		\texttt{pop\_size-3000+elite-30 		}	 &	161.26 &	 222.73 &	  3.17 sec \\
		\texttt{pop\_size-3000+elite-10 		}	 &	228.41 &	 229.67 &	  3.04 sec \\
		\texttt{iter-200 						}		&              1356.23 &	1356.23 &	  1.09 sec \\
		\texttt{iter-50 						}		 &              1290.16 &	1290.59 &	  0.69 sec \\
		\texttt{iter-30 						}		 &              1393.38 &	1394.41 &	  0.66 sec \\
		\texttt{iter-200+elite-30 				}	&              1229.94 &	1230.02 &	  1.11 sec \\
		\texttt{iter-200+pop\_size-3000+elite-30} 	 &	101.37 &	 103.35 &	  6.42 sec \\
		\bottomrule
	\end{tabular}
	\captionof{table}{Rastrigin CMA-ES performance}
	\label{tab:cmaes-performance-r}
\end{figure}


In Figure \ref{fig:cmaes-r-fitness/baseline} is plotted the best and the worse fitness for each generation (averaged over 3 runs) for the baseline model. Below, in Figures \ref{fig:cmaes-r-fitness/3000} and \ref{fig:cmaes-r-fitness/3000iter} are plotted the fitness of the models that perform better with the rastrigin function.

\begin{figure}[H]
	\centering
	\begin{minipage}[b]{.6\textwidth}
		\includegraphics[width=\linewidth]{../src/out/cma_es/baseline/rastrigin/fitness.pdf}	
	\end{minipage}
	\caption{Rastrigin fitness \texttt{baseline}}
	\label{fig:cmaes-r-fitness/baseline}

	\begin{minipage}[b]{.6\textwidth}
		\includegraphics[width=\linewidth]{../src/out/cma_es/pop_size-3000/rastrigin/fitness.pdf}	
	\end{minipage}
	\caption{Rastrigin fitness \texttt{pop\_size-3000}}
	\label{fig:cmaes-r-fitness/3000}
	
		\begin{minipage}[b]{.6\textwidth}
		\includegraphics[width=\linewidth]{../src/out/cma_es/pop_size-3000+iter-200+elite-30/rastrigin/fitness.pdf}	
	\end{minipage}
	\caption{Rastrigin fitness \texttt{pop\_size-3000+iter-200+elite-30}}
	\label{fig:cmaes-r-fitness/3000iter}
\end{figure}

In the following Figures is shown the effect of executing the algorithm on the experiment \texttt{pop\_size-3000+iter-200+elite-30}. 
\begin{figure}[H]
	\begin{minipage}[b]{.3\textwidth}
		\centering
		\includegraphics[width=\linewidth]{../src/out/cma_es/pop_size-3000+iter-200+elite-30/rastrigin/cma_es-generation-contour-0.pdf}	
	\end{minipage}
	~
	\begin{minipage}[b]{.3\textwidth}
		\centering
		\includegraphics[width=\linewidth]{../src/out/cma_es/pop_size-3000+iter-200+elite-30/rastrigin/cma_es-generation-contour-20.pdf}	
	\end{minipage}
	~
	\begin{minipage}[b]{.3\textwidth}
		\centering
		\includegraphics[width=\linewidth]{../src/out/cma_es/pop_size-3000+iter-200+elite-30/rastrigin/cma_es-generation-contour-40.pdf}	
	\end{minipage}
	
	\begin{minipage}[b]{.3\textwidth}
		\centering
		\includegraphics[width=\linewidth]{../src/out/cma_es/pop_size-3000+iter-200+elite-30/rastrigin/cma_es-generation-contour-60.pdf}	
	\end{minipage}
	~
	\begin{minipage}[b]{.3\textwidth}
		\centering
		\includegraphics[width=\linewidth]{../src/out/cma_es/pop_size-3000+iter-200+elite-30/rastrigin/cma_es-generation-contour-80.pdf}	
	\end{minipage}
	~
	\begin{minipage}[b]{.3\textwidth}
		\centering
		\includegraphics[width=\linewidth]{../src/out/cma_es/pop_size-3000+iter-200+elite-30/rastrigin/cma_es-generation-contour-100.pdf}	
	\end{minipage}
	
	\begin{minipage}[b]{.3\textwidth}
		\centering
		\includegraphics[width=\linewidth]{../src/out/cma_es/pop_size-3000+iter-200+elite-30/rastrigin/cma_es-generation-contour-120.pdf}	
	\end{minipage}
	~
	\begin{minipage}[b]{.3\textwidth}
		\centering
		\includegraphics[width=\linewidth]{../src/out/cma_es/pop_size-3000+iter-200+elite-30/rastrigin/cma_es-generation-contour-140.pdf}	
	\end{minipage}
	~
	\begin{minipage}[b]{.3\textwidth}
		\centering
		\includegraphics[width=\linewidth]{../src/out/cma_es/pop_size-3000+iter-200+elite-30/rastrigin/cma_es-generation-contour-160.pdf}	
	\end{minipage}
	
\end{figure}

For the sphere function, 40 generations are enough to obtain a solution close enough to the global optimum, as shown for the experiment \texttt{pop\_size-3000}.

For the rastrigin function and the parameter tested, it was not possible to converge to a global optimum. 