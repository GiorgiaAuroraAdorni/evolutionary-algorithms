\section{Natural Evolution Strategy (NES)}

The second algorithm implemented is the \textbf{Natural Evolution Strategy (NES)}. 

As for the previous method, different execution of the algorithm has been performed. The first experiment is called baseline, and it uses the same parameters as for the following algorithms. After that, were made some attempts of improving the performances trying different population size and elite set ratio, and also with different number of generations. In all the experiments the domain size is fixed to 100 dimension.

For all the experiments, the initial population parameters are initialised reasonably far from the global optimum, in particular, the mean is uniformly sample in the range [-5, 5] and the covariance matrix is initialised as a diagonal matrix with points uniformly sampled in the range [4, 5].

In Table \ref{tab:nes-param} are shown the different combinations of parameter used for this algorithm.
\begin{figure}[htb]
	\centering
	
	\begin{tabular}{lccc}
		\toprule
		\textbf{experiment}  & \textbf{population} & \textbf{learning rate} &
		\textbf{generations} \\
		\midrule
		\texttt{baseline 						} & 100 	& 1e-2 	& 100\\
		\texttt{pop\_size-1000 					} & 1000 	& 1e-2 	& 100\\
		\texttt{pop\_size-3000 					} & 3000 	& 1e-2 	& 100\\
		\texttt{pop\_size-5000 					} & 5000 	& 1e-2 	& 100\\
		\texttt{lr-001 							} & 100 	& 1e-3 	& 100\\
		\texttt{lr-0001	 						} & 100 	& 1e-4 	& 100\\
		\texttt{lr-00001	 					} & 100 	& 1e-5 	& 100\\
		\texttt{pop\_size-1000+lr-001 			} & 1000 	& 1e-3 	& 100\\
		\texttt{pop\_size-1000+lr-0001 			} & 1000 	& 1e-4 	& 100\\
		\texttt{pop\_size-5000+lr-001 			} & 5000 	& 1e-3 	& 100\\
		\texttt{pop\_size-5000+lr-0001 			} & 5000 	& 1e-4 	& 100\\
		\texttt{iter-2000 						} & 100 	& 1e-2 	& 2000\\
		\texttt{iter-5000 						} & 100 	& 1e-2 	& 5000\\
		\texttt{iter-2000+pop-5000 	}			 & 5000 	& 1e-2 	& 2000\\
		\texttt{iter-5000+pop-5000 	}			 & 5000 	& 1e-2 	& 5000\\
		\texttt{iter-2000+pop-5000+lr-001 }		 & 5000 	& 1e-3 	& 5000\\
		\texttt{iter-2000+pop-5000+lr-0001 }	 & 5000 	& 1e-4 	& 5000\\
		\bottomrule
	\end{tabular}
	\captionof{table}{NES parameters}
	\label{tab:nes-param}
\end{figure}

The algorithm has been run 3 times for both of test functions. 
In order to evaluate the performance, for each pair of experiment and test function, the best and the worse fitness for each generation (averaged over 3 runs) has been plotted. 
\bigskip

In Tables \ref{tab:cem-performance-s} and \ref{tab:cem-performance-r} are summarised the results. 
\bigskip

\begin{figure}[htb]
	\centering
	
	\begin{tabular}{lccc}
		\toprule
		\textbf{experiment} & \textbf{best fitness} & \textbf{worse fitness} & \textbf{avg run time} \\
		\midrule
		\texttt{baseline 						}  &	   Err &	    Err &	       Err \\
		\texttt{pop\_size-1000 					}     &	   Err &	    Err &	       Err \\
		\texttt{pop\_size-3000 					}     &	 31.77 &	  55.11 &	  1.37 sec \\
		\texttt{pop\_size-5000 					}     &	 37.28 &	 103.57 &	  4.46 sec \\
		\texttt{lr-001 							}    &	   Err &	    Err &	       Err \\
		\texttt{lr-0001	 						}   &	   Err &	    Err &	       Err \\
		\texttt{lr-00001	 					}  &	   Err &	    Err &	       Err \\
		\texttt{pop\_size-1000+lr-001 			}   &	326.19 &	  831.2 &	  1.28 sec \\
		\texttt{pop\_size-1000+lr-0001 			}   &              1154.99 &	3024.52 &	  1.33 sec \\
		\texttt{pop\_size-5000+lr-001 			}   &	247.58 &	  702.4 &	   4.9 sec \\
		\texttt{pop\_size-5000+lr-0001 			}   &              1090.84 &	2837.48 &	  4.67 sec \\
		\texttt{iter-2000 						}   &	   Err &	    Err &	       Err \\
		\texttt{iter-5000 						}   &	   Err &	    Err &	       Err \\
		\texttt{iter-2000+pop-5000 	}			    &	   0.5 &	   1.44 &	 26.06 sec \\
		\texttt{iter-5000+pop-5000 	}			    &	  0.21 &	   0.55 &	 60.59 sec \\
		\texttt{iter-2000+pop-5000+lr-001 }		    &	 14.39 &	  41.89 &	 81.64 sec \\
		\texttt{iter-2000+pop-5000+lr-0001 }	    &	132.99 &	 383.86 &	 84.32 sec \\
		\bottomrule
	\end{tabular}
	\captionof{table}{Sphere NES performance}
	\label{tab:nes-performance-s}
\end{figure}

For the sphere function, it is possible to see that for many combinations of parameters it is not possible to execute the algorithm. This is because a \texttt{ValueError} is often returned due to the transformation of the covariance matrix into a non-positive one.
Therefore, by increasing the population size enough and the number of generations, the algorithm is able to converge.
With the presented parameters, using too low a learning rate reduces the algorithm's convergence speed.
\bigskip

In Figure \ref{fig:new-s-fitness/2000}, \ref{fig:new-s-fitness/5000} and \ref{fig:new-s-fitness/lr} are plotted the best and the worse fitness for each generation (averaged over 3 runs) for the models that perform better with the sphere function.

 \begin{figure}[H]
	\centering
	\begin{minipage}[b]{.6\textwidth}
		\includegraphics[width=\linewidth]{../src/out/nes/iteration-2000+pop_size-5000/sphere/fitness.pdf}	
	\end{minipage}
	\caption{Sphere fitness \texttt{iteration-2000+pop\_size-5000}}
	\label{fig:new-s-fitness/2000}
\end{figure}

\begin{figure}[H]
\centering
	\begin{minipage}[b]{.6\textwidth}
		\includegraphics[width=\linewidth]{../src/out/nes/iteration-5000+pop_size-5000/sphere/fitness.pdf}	
	\end{minipage}
	\caption{Sphere fitness \texttt{iteration-5000+pop\_size-5000}}
	\label{fig:new-s-fitness/5000}

	\begin{minipage}[b]{.6\textwidth}
		\includegraphics[width=\linewidth]{../src/out/nes/iteration-2000+pop_size-5000+lr-001/sphere/fitness.pdf}	
	\end{minipage}
	\caption{Sphere fitness \texttt{iteration-2000+pop\_size-5000+lr-001}}
	\label{fig:new-s-fitness/lr}
\end{figure}

For the rastrigin function, it is fundamental to increase the number of generations to allow the algorithm to converge.
With the tested parameters, it was not possible to converge the algorithm to a global optimum.

In Figure \ref{fig:nes-r-fitness/001} and \ref{fig:nes-r-fitness/0001} are plotted the best and the worse fitness for each generation (averaged over 3 runs) of two models performed.

\begin{figure}[htb]
	\centering
	
	\begin{tabular}{lccc}
		\toprule
		\textbf{experiment} & \textbf{best fitness} & \textbf{worse fitness} & \textbf{avg run time} \\
		\midrule
		\texttt{baseline 						} &	   Err &	    Err &	       Err \\
		\texttt{pop\_size-1000 					}    &	   Err &	    Err &	       Err \\
		\texttt{pop\_size-3000 					}    &	   Err &	    Err &	       Err \\
		\texttt{pop\_size-5000 					}    &	   Err &	    Err &	       Err \\
		\texttt{lr-001 							}   &	   Err &	    Err &	       Err \\
		\texttt{lr-0001	 						}  &	   Err &	    Err &	       Err \\
		\texttt{lr-00001	 					} &	   Err &	    Err &	       Err \\
		\texttt{pop\_size-1000+lr-001 			} &              1358.28 &	2154.99 &	  1.74 sec \\
		\texttt{pop\_size-1000+lr-0001 			} &              2169.83 &	4029.91 &	  1.94 sec \\
		\texttt{pop\_size-5000+lr-001 			} &              1142.55 &	1834.66 &	  6.05 sec \\
		\texttt{pop\_size-5000+lr-0001 			} &              2000.94 &	4007.66 &	  5.74 sec \\
		\texttt{iter-2000 						}  &	   Err &	    Err &	       Err \\
		\texttt{iter-5000 						}  &	   Err &	    Err &	       Err \\
		\texttt{iter-2000+pop-5000 	}			  &	   Err &	    Err &	       Err \\
		\texttt{iter-5000+pop-5000 	}			  &	   Err &	    Err &	       Err \\
		\texttt{iter-2000+pop-5000+lr-001 }		  &	764.96 &	1279.41 &	103.81 sec \\
		\texttt{iter-2000+pop-5000+lr-0001 }	  &	955.97 &	1558.37 &	106.25 sec \\
		\bottomrule
	\end{tabular}
	\captionof{table}{Rastrigin NES performance}
	\label{tab:nes-performance-r}
\end{figure}

\bigskip
For the sphere function, 2000 generations are enough to obtain a solution close enough to the global optimum, as shown for the experiment \texttt{iteration-2000+pop\_size-5000}.

For the rastrigin function are probably needed more than 5000 generations to obtain a solution close enough to the global optimum.

\begin{figure}[H]
	\centering
	\begin{minipage}[b]{.6\textwidth}
		\includegraphics[width=\linewidth]{../src/out/nes/iteration-2000+pop_size-5000+lr-001/rastrigin/fitness.pdf}	
	\end{minipage}
	\caption{Rastrigin fitness \texttt{iteration-2000+pop\_size-5000+lr-001000}}
	\label{fig:nes-r-fitness/001}
\end{figure}
\begin{figure}[H]
	\centering
	\begin{minipage}[b]{.6\textwidth}
		\includegraphics[width=\linewidth]{../src/out/nes/iteration-2000+pop_size-5000+lr-0001/rastrigin/fitness.pdf}	
	\end{minipage}
	\caption{Rastrigin fitness \texttt{iteration-2000+pop\_size-5000+lr-0001}}
	\label{fig:nes-r-fitness/0001}
\end{figure}
