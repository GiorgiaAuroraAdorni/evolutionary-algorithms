\documentclass[a4paper,12pt]{article} % This defines the style of your paper

\usepackage[top = 2.5cm, bottom = 2.5cm, left = 2.5cm, right = 2.5cm]{geometry} 
\usepackage[utf8]{inputenc} %utf8 % lettere accentate da tastiera
\usepackage[english]{babel} % lingua del documento
\usepackage[T1]{fontenc} % codifica dei font

\usepackage{multirow} % Multirow is for tables with multiple rows within one 
%cell.
\usepackage{booktabs} % For even nicer tables.

\usepackage{graphicx} 

\usepackage{setspace}
\setlength{\parindent}{0in}

\usepackage{float}

\usepackage{fancyhdr}

\usepackage{caption}
\usepackage{amssymb}
\usepackage{amsmath}
\usepackage{mathtools}
\usepackage{color}

\usepackage[hidelinks]{hyperref}
\usepackage{csquotes}
\usepackage{subfigure}

\usepackage{ifxetex,ifluatex}
\usepackage{etoolbox}
\usepackage[svgnames]{xcolor}

\usepackage{tikz}

\usepackage{framed}

 \newcommand*\quotefont{\fontfamily{LinuxLibertineT-LF}} % selects Libertine as 
 %the quote font


\newcommand*\quotesize{40} % if quote size changes, need a way to make shifts 
%relative
% Make commands for the quotes
\newcommand*{\openquote}
{\tikz[remember picture,overlay,xshift=-4ex,yshift=-1ex]
	\node (OQ) 
	{\quotefont\fontsize{\quotesize}{\quotesize}\selectfont``};\kern0pt}

\newcommand*{\closequote}[1]
{\tikz[remember picture,overlay,xshift=4ex,yshift=-1ex]
	\node (CQ) {\quotefont\fontsize{\quotesize}{\quotesize}\selectfont''};}

% select a colour for the shading
\colorlet{shadecolor}{WhiteSmoke}

\newcommand*\shadedauthorformat{\emph} % define format for the author argument

% Now a command to allow left, right and centre alignment of the author
\newcommand*\authoralign[1]{%
	\if#1l
	\def\authorfill{}\def\quotefill{\hfill}
	\else
	\if#1r
	\def\authorfill{\hfill}\def\quotefill{}
	\else
	\if#1c
	\gdef\authorfill{\hfill}\def\quotefill{\hfill}
	\else\typeout{Invalid option}
	\fi
	\fi
	\fi}
% wrap everything in its own environment which takes one argument (author) and 
%one optional argument
% specifying the alignment [l, r or c]
%
\newenvironment{shadequote}[2][l]%
{\authoralign{#1}
	\ifblank{#2}
	{\def\shadequoteauthor{}\def\yshift{-2ex}\def\quotefill{\hfill}}
	{\def\shadequoteauthor{\par\authorfill\shadedauthorformat{#2}}\def\yshift{2ex}}
	\begin{snugshade}\begin{quote}\openquote}
		{\shadequoteauthor\quotefill\closequote{\yshift}\end{quote}\end{snugshade}}

\newcommand{\footref}[1]{%
	$^{\ref{#1}}$%
}

\pagestyle{fancy}

\setlength\parindent{24pt}

\fancyhf{}

\lhead{\footnotesize Machine Learning: Assignment 5}

\rhead{\footnotesize Giorgia Adorni}

\cfoot{\footnotesize \thepage} 

\begin{document}
%	\thispagestyle{empty}  
	\noindent{
	\begin{tabular}{p{15cm}} 
		{\large \bf Machine Learning} \\
		Università della Svizzera Italiana \\ Faculty of Informatics \\ \today  \\
		\hline
		\\
	\end{tabular} 
	
	\vspace*{0.3cm} 
	
	\begin{center}
		{\Large \bf Assignment 5: Evolutionary Algorithms}
		\vspace{2mm}
		
		{\bf Giorgia Adorni (giorgia.adorni@usi.ch)}
		
	\end{center}  
}
	\vspace{0.4cm}

	%%%%%%%%%%%%%%%%%%%%%%%%%%%%%%%%%%%%%%%%%%%%%%%%
	%%%%%%%%%%%%%%%%%%%%%%%%%%%%%%%%%%%%%%%%%%%%%%%%

\section{Introduction}
In this assignment, we are going to implement a few evolutionary algorithms against test functions and analyze the characteristics of different algorithms.
Let $f : \mathbb{R}^n \rightarrow \mathbb{R}$ be a test function with n-dimensional domain.

\noindent\textbf{Sphere function}:
	\begin{equation}
		f(\overline{x}) = \sum_{i=1}^nx_i^2 \mbox{.}
	\end{equation}
\textbf{Rastrigin functionction}:
	\begin{equation}
	f(\overline{x}) = An + \sum_{i=1}^n [x_i^2-A \cos(2\pi x_i)]
	\end{equation}
	where $A = 10$. The search domain is constraint as $x_i \in [-5, 5]$.

\section{Test Functions}

In Figure \ref{fig:s} and \ref{fig:r} are shown the 2D contour-plot of 2-dimensional test functions.

\begin{figure}[htb]
	\begin{minipage}[b]{.49\textwidth}
		\centering
		\includegraphics[width=\linewidth]{../src/out/sphere_test.pdf}	
		\caption{2D contour plot of 2D Sphere}
		\label{fig:s}
	\end{minipage}
	~
	\begin{minipage}[b]{.49\textwidth}
		\centering
		\includegraphics[width=\linewidth]{../src/out/rastrigin_test.pdf}	
		\caption{2D contour plot of 2D Rastrigin}
		\label{fig:r}
	\end{minipage}
\end{figure}

\begin{figure}[htb]
	\begin{minipage}[b]{.49\textwidth}
		\centering
		\includegraphics[width=\linewidth]{../src/out/sphere_test-eval.pdf}	
		\caption{Sphere test function evaluation}
		\label{fig:s-eval}
	\end{minipage}
	~
	\begin{minipage}[b]{.49\textwidth}
		\centering
		\includegraphics[width=\linewidth]{../src/out/rastrigin_test-eval.pdf}	
		\caption{Rastrigin test function evaluation}
		\label{fig:r-eval}
	\end{minipage}
\end{figure}

For each test function, 100 points has been uniformly sampled in the domain and evaluated with the test function. This points are shown in Figure \ref{fig:s-eval} and \ref{fig:r-eval}. 
Filling the contours and showing the colour-map it is easier to see which are the regions of the global optimum. All the points with value close to 0, the ones in purple, belong to the global optimal region.


For the sphere, the points are concentrated in the central region of the figure. 

For the rastrigin function evaluate the samples is many more complicated, since the function has many local minima. However, the global optimum is located in the centre of plot. For this reason, it will be seen later that finding the global optimum of this function will be difficult.
 
\section{Cross-Entropy Method (CEM)}

The first algorithm implemented is the \textbf{Cross-Entropy Method (CEM)}.

The initial population parameters are initialised reasonably far from the global optimum, in particular, the mean is uniformly sample in the range [-5, 5] and the variance is uniformly sample in the range [0, 1].

The algorithm has been run for the baseline model, and then trying different population size and elite set ratio, and also with different number of generations.
In Table \ref{tab:cem-param} are shown the different combinations of parameter used.

\begin{figure}[htb]
	\centering
	
	\begin{tabular}{lcccc}
		\toprule
		\textbf{experiment} & \textbf{domain} & \textbf{population} & \textbf{elite} &
		\textbf{generations} \\
		\midrule
		\texttt{baseline 						}	& 100 & 30 		& 0.20 	& 100\\
		\texttt{pop\_size-100 					}	& 100 & 100 	& 0.20 	& 100\\
		\texttt{pop\_size-1000 					}	& 100 & 1000 	& 0.20 	& 100\\
		\texttt{elite-30 						}	& 100 & 30 		& 0.30 	& 100\\
		\texttt{elite-10 						}	& 100 & 30 		& 0.10 	& 100\\
		\texttt{pop\_size-100+elite-30 			}	& 100 & 100 	& 0.30 	& 100\\
		\texttt{pop\_size-100+elite-10 			}	& 100 & 100 	& 0.10 	& 100\\
		\texttt{pop\_size-1000+elite-30 		}	& 100 & 1000 	& 0.30 	& 100\\
		\texttt{pop\_size-1000+elite-10 		}	& 100 & 1000 	& 0.10 	& 100\\
		\texttt{iter-50 						}	& 100 & 30 		& 0.20 	& 50\\
		\texttt{iter-30 						}	& 100 & 30 		& 0.20 	& 30\\
		\texttt{iter-30+elite-10 				}	& 100 & 30 		& 0.10 	& 30\\
		\texttt{iter-50+elite-10 				}	& 100 & 30 	& 0.10 	& 50\\
		\texttt{iter-200+pop\_size-1000+elite-30} 	& 100 & 1000 	& 0.30 	& 200\\	
		\bottomrule
	\end{tabular}
	\captionof{table}{CEM parameters}
	\label{tab:cem-param}
\end{figure}

For all the experiments, the algorithm has been run 3 times for both of test functions. In order to evaluate the performance, for each pair of experiment and test function, the best and the worse fitness for each generation (averaged over 3 runs) has been plotted. 

In Tables \ref{tab:cem-performance-s} and \ref{tab:cem-performance-r} are summarised the results.

\begin{figure}[htb]
	\centering
	
	\begin{tabular}{lccc}
		\toprule
		\textbf{experiment} & \textbf{best fitness} & \textbf{worse fitness} & \textbf{avg run time} \\
		\midrule
		\texttt{baseline 						}	& 408.15 &                408.15 &                 0.36 sec \\
		\texttt{pop\_size-100 					}	&   76.19 &                 76.31 &                 0.41 sec \\
		\texttt{pop\_size-1000 					}	&     0.0 &                   0.0 &                 0.89 sec \\
		\texttt{elite-30 						}	& 353.41 &                353.41 &                 0.34 sec \\
		\texttt{elite-10 						}	& 549.93 &                549.93 &                 0.34 sec \\
		\texttt{pop\_size-100+elite-30 			}	&   39.73 &                 40.13 &                 0.38 sec \\
		\texttt{pop\_size-100+elite-10 			}	&  168.58 &                168.58 &                 0.38 sec \\
		\texttt{pop\_size-1000+elite-30 		}	&     0.0 &                   0.0 &                 0.88 sec \\
		\texttt{pop\_size-1000+elite-10 		}	&     0.0 &                   0.0 &                 0.88 sec \\
		\texttt{iter-50 						}	& 365.66 &                365.69 &                 0.33 sec \\
		\texttt{iter-30 						}	& 405.81 &                407.39 &                 0.32 sec \\
		\texttt{iter-30+elite-10 				}	& 606.47 &                606.48 &                 0.33 sec \\
		\texttt{iter-50+elite-10 				}	& 591.32 &                591.32 &                 0.43 sec \\
		\texttt{iter-200+pop\_size-1000+elite-30} 	&     0.0 &                   0.0 &                 1.39 sec \\	
		\bottomrule
	\end{tabular}
	\captionof{table}{Sphere CEM performance}
	\label{tab:cem-performance-s}
\end{figure}

\begin{figure}[htb]
	\centering
	
	\begin{tabular}{lccc}
		\toprule
		\textbf{experiment} & \textbf{best fitness} & \textbf{worse fitness} & \textbf{avg run time} \\
		\midrule
		\texttt{baseline 						}	&  917.88 &                917.88 &                 0.56 sec \\
		\texttt{pop\_size-100 					}	&  372.42 &                372.93 &                 0.62 sec \\
		\texttt{pop\_size-1000 					}	&  340.85 &                 603.9 &                 1.27 sec \\
		\texttt{elite-30 						}	&  873.88 &                 873.9 &                 0.55 sec \\
		\texttt{elite-10 						}	&  1268.29 &               1268.29 &                 0.55 sec \\
		\texttt{pop\_size-100+elite-30 			}	&  314.37 &                316.53 &                  0.6 sec \\
		\texttt{pop\_size-100+elite-10 			}	&  559.49 &                 559.5 &                  0.6 sec \\
		\texttt{pop\_size-1000+elite-30 		}	&  604.86 &               1016.53 &                 1.27 sec \\
		\texttt{pop\_size-1000+elite-10 		}	&  94.1 &                 147.3 &                 1.24 sec \\
		\texttt{iter-50 						}	&  916.63 &                917.04 &                 0.53 sec \\
		\texttt{iter-30 						}	&  975.49 &                997.92 &                 0.53 sec \\
		\texttt{iter-30+elite-10 				}	& 1228.54 &               1228.58 &                 0.54 sec \\
		\texttt{iter-50+elite-10 				}	& 1276.75 &               1276.75 &                  0.7 sec \\
		\texttt{iter-200+pop\_size-1000+elite-30} 	&   33.64 &                 35.85 &                 1.97 sec \\
		\bottomrule
	\end{tabular}
	\captionof{table}{Rastrigin CEM performance}
	\label{tab:cem-performance-r}
\end{figure}


(c) Try . What is the minimum number of gener- ations that you can obtain a solution close enough to the global optimum?

\section{Natural Evolution Strategy (NES)}

The second algorithm implemented is the \textbf{Natural Evolution Strategy (NES)}. 

As for the previous method, different execution of the algorithm has been performed. The first experiment is called baseline, and it uses the same parameters as for the following algorithms. After that, were made some attempts of improving the performances trying different population size and elite set ratio, and also with different number of generations. In all the experiments the domain size is fixed to 100 dimension.

For all the experiments, the initial population parameters are initialised reasonably far from the global optimum, in particular, the mean is uniformly sample in the range [-5, 5] and the covariance matrix is initialised as a diagonal matrix with points uniformly sampled in the range [4, 5].

In Table \ref{tab:nes-param} are shown the different combinations of parameter used for this algorithm.
\begin{figure}[htb]
	\centering
	
	\begin{tabular}{lccc}
		\toprule
		\textbf{experiment}  & \textbf{population} & \textbf{learning rate} &
		\textbf{generations} \\
		\midrule
		\texttt{baseline 						} & 100 	& 1e-2 	& 100\\
		\texttt{pop\_size-1000 					} & 1000 	& 1e-2 	& 100\\
		\texttt{pop\_size-3000 					} & 3000 	& 1e-2 	& 100\\
		\texttt{pop\_size-5000 					} & 5000 	& 1e-2 	& 100\\
		\texttt{lr-001 							} & 100 	& 1e-3 	& 100\\
		\texttt{lr-0001	 						} & 100 	& 1e-4 	& 100\\
		\texttt{lr-00001	 					} & 100 	& 1e-5 	& 100\\
		\texttt{pop\_size-1000+lr-001 			} & 1000 	& 1e-3 	& 100\\
		\texttt{pop\_size-1000+lr-0001 			} & 1000 	& 1e-4 	& 100\\
		\texttt{pop\_size-5000+lr-001 			} & 5000 	& 1e-3 	& 100\\
		\texttt{pop\_size-5000+lr-0001 			} & 5000 	& 1e-4 	& 100\\
		\texttt{iter-2000 						} & 100 	& 1e-2 	& 2000\\
		\texttt{iter-5000 						} & 100 	& 1e-2 	& 5000\\
		\texttt{iter-2000+pop-5000 	}			 & 5000 	& 1e-2 	& 2000\\
		\texttt{iter-5000+pop-5000 	}			 & 5000 	& 1e-2 	& 5000\\
		\texttt{iter-2000+pop-5000+lr-001 }		 & 5000 	& 1e-3 	& 5000\\
		\texttt{iter-2000+pop-5000+lr-0001 }	 & 5000 	& 1e-4 	& 5000\\
		\bottomrule
	\end{tabular}
	\captionof{table}{NES parameters}
	\label{tab:nes-param}
\end{figure}

The algorithm has been run 3 times for both of test functions. 
In order to evaluate the performance, for each pair of experiment and test function, the best and the worse fitness for each generation (averaged over 3 runs) has been plotted. 
\bigskip

In Tables \ref{tab:cem-performance-s} and \ref{tab:cem-performance-r} are summarised the results. 
\bigskip

\begin{figure}[htb]
	\centering
	
	\begin{tabular}{lccc}
		\toprule
		\textbf{experiment} & \textbf{best fitness} & \textbf{worse fitness} & \textbf{avg run time} \\
		\midrule
		\texttt{baseline 						}  &	   Err &	    Err &	       Err \\
		\texttt{pop\_size-1000 					}     &	   Err &	    Err &	       Err \\
		\texttt{pop\_size-3000 					}     &	 31.77 &	  55.11 &	  1.37 sec \\
		\texttt{pop\_size-5000 					}     &	 37.28 &	 103.57 &	  4.46 sec \\
		\texttt{lr-001 							}    &	   Err &	    Err &	       Err \\
		\texttt{lr-0001	 						}   &	   Err &	    Err &	       Err \\
		\texttt{lr-00001	 					}  &	   Err &	    Err &	       Err \\
		\texttt{pop\_size-1000+lr-001 			}   &	326.19 &	  831.2 &	  1.28 sec \\
		\texttt{pop\_size-1000+lr-0001 			}   &              1154.99 &	3024.52 &	  1.33 sec \\
		\texttt{pop\_size-5000+lr-001 			}   &	247.58 &	  702.4 &	   4.9 sec \\
		\texttt{pop\_size-5000+lr-0001 			}   &              1090.84 &	2837.48 &	  4.67 sec \\
		\texttt{iter-2000 						}   &	   Err &	    Err &	       Err \\
		\texttt{iter-5000 						}   &	   Err &	    Err &	       Err \\
		\texttt{iter-2000+pop-5000 	}			    &	   0.5 &	   1.44 &	 26.06 sec \\
		\texttt{iter-5000+pop-5000 	}			    &	  0.21 &	   0.55 &	 60.59 sec \\
		\texttt{iter-2000+pop-5000+lr-001 }		    &	 14.39 &	  41.89 &	 81.64 sec \\
		\texttt{iter-2000+pop-5000+lr-0001 }	    &	132.99 &	 383.86 &	 84.32 sec \\
		\bottomrule
	\end{tabular}
	\captionof{table}{Sphere NES performance}
	\label{tab:nes-performance-s}
\end{figure}

For the sphere function, it is possible to see that for many combinations of parameters it is not possible to execute the algorithm. This is because a \texttt{ValueError} is often returned due to the transformation of the covariance matrix into a non-positive one.
Therefore, by increasing the population size enough and the number of generations, the algorithm is able to converge.
With the presented parameters, using too low a learning rate reduces the algorithm's convergence speed.
\bigskip

In Figure \ref{fig:new-s-fitness/2000}, \ref{fig:new-s-fitness/5000} and \ref{fig:new-s-fitness/lr} are plotted the best and the worse fitness for each generation (averaged over 3 runs) for the models that perform better with the sphere function.

 \begin{figure}[H]
	\centering
	\begin{minipage}[b]{.6\textwidth}
		\includegraphics[width=\linewidth]{../src/out/nes/iteration-2000+pop_size-5000/sphere/fitness.pdf}	
	\end{minipage}
	\caption{Sphere fitness \texttt{iteration-2000+pop\_size-5000}}
	\label{fig:new-s-fitness/2000}
\end{figure}

\begin{figure}[H]
\centering
	\begin{minipage}[b]{.6\textwidth}
		\includegraphics[width=\linewidth]{../src/out/nes/iteration-5000+pop_size-5000/sphere/fitness.pdf}	
	\end{minipage}
	\caption{Sphere fitness \texttt{iteration-5000+pop\_size-5000}}
	\label{fig:new-s-fitness/5000}

	\begin{minipage}[b]{.6\textwidth}
		\includegraphics[width=\linewidth]{../src/out/nes/iteration-2000+pop_size-5000+lr-001/sphere/fitness.pdf}	
	\end{minipage}
	\caption{Sphere fitness \texttt{iteration-2000+pop\_size-5000+lr-001}}
	\label{fig:new-s-fitness/lr}
\end{figure}

For the rastrigin function, it is fundamental to increase the number of generations to allow the algorithm to converge.
With the tested parameters, it was not possible to converge the algorithm to a global optimum.

In Figure \ref{fig:nes-r-fitness/001} and \ref{fig:nes-r-fitness/0001} are plotted the best and the worse fitness for each generation (averaged over 3 runs) of two models performed.

\begin{figure}[htb]
	\centering
	
	\begin{tabular}{lccc}
		\toprule
		\textbf{experiment} & \textbf{best fitness} & \textbf{worse fitness} & \textbf{avg run time} \\
		\midrule
		\texttt{baseline 						} &	   Err &	    Err &	       Err \\
		\texttt{pop\_size-1000 					}    &	   Err &	    Err &	       Err \\
		\texttt{pop\_size-3000 					}    &	   Err &	    Err &	       Err \\
		\texttt{pop\_size-5000 					}    &	   Err &	    Err &	       Err \\
		\texttt{lr-001 							}   &	   Err &	    Err &	       Err \\
		\texttt{lr-0001	 						}  &	   Err &	    Err &	       Err \\
		\texttt{lr-00001	 					} &	   Err &	    Err &	       Err \\
		\texttt{pop\_size-1000+lr-001 			} &              1358.28 &	2154.99 &	  1.74 sec \\
		\texttt{pop\_size-1000+lr-0001 			} &              2169.83 &	4029.91 &	  1.94 sec \\
		\texttt{pop\_size-5000+lr-001 			} &              1142.55 &	1834.66 &	  6.05 sec \\
		\texttt{pop\_size-5000+lr-0001 			} &              2000.94 &	4007.66 &	  5.74 sec \\
		\texttt{iter-2000 						}  &	   Err &	    Err &	       Err \\
		\texttt{iter-5000 						}  &	   Err &	    Err &	       Err \\
		\texttt{iter-2000+pop-5000 	}			  &	   Err &	    Err &	       Err \\
		\texttt{iter-5000+pop-5000 	}			  &	   Err &	    Err &	       Err \\
		\texttt{iter-2000+pop-5000+lr-001 }		  &	764.96 &	1279.41 &	103.81 sec \\
		\texttt{iter-2000+pop-5000+lr-0001 }	  &	955.97 &	1558.37 &	106.25 sec \\
		\bottomrule
	\end{tabular}
	\captionof{table}{Rastrigin NES performance}
	\label{tab:nes-performance-r}
\end{figure}

\begin{figure}[H]
	\centering
	\begin{minipage}[b]{.6\textwidth}
		\includegraphics[width=\linewidth]{../src/out/nes/iteration-2000+pop_size-5000+lr-001/rastrigin/fitness.pdf}	
	\end{minipage}
	\caption{Sphere fitness \texttt{iteration-2000+pop\_size-5000+lr-001000}}
	\label{fig:nes-r-fitness/001}
\end{figure}
\begin{figure}[H]
	\centering
	\begin{minipage}[b]{.6\textwidth}
		\includegraphics[width=\linewidth]{../src/out/nes/iteration-2000+pop_size-5000+lr-0001/rastrigin/fitness.pdf}	
	\end{minipage}
	\caption{Sphere fitness \texttt{iteration-2000+pop\_size-5000+lr-0001}}
	\label{fig:nes-r-fitness/0001}
\end{figure}
\section{Covariance Matrix Adaptation Evolution Strategy \\(CMA-ES)}

\begin{figure}[htb]
	\centering
	
	\begin{tabular}{lcccc}
		\toprule
		\textbf{experiment} & \textbf{domain} & \textbf{population} & \textbf{elite} &
		\textbf{generations} \\
		\midrule
		\texttt{baseline 							}	& 100 & 30 	& 0.20 	& 100\\
		\texttt{pop\_size-100 						}	& 100 & 100 	& 0.20 	& 100\\
		\texttt{pop\_size-1000 						}	& 100 & 1000 	& 0.20 	& 100\\
		\texttt{elite-30 							}	& 100 & 30 	& 0.30 	& 100\\
		\texttt{elite-10 							}	& 100 & 30 	& 0.10 	& 100\\
		\texttt{pop\_size-100+elite-30 				}	& 100 & 100 	& 0.30 	& 100\\
		\texttt{pop\_size-100+elite-10 				}	& 100 & 100 	& 0.10 	& 100\\
		\texttt{pop\_size-1000+elite-30 				}	& 100 & 1000 	& 0.30 	& 100\\
		\texttt{pop\_size-1000+elite-10 				}	& 100 & 1000 	& 0.10 	& 100\\
		\texttt{iter-50 						}	& 100 & 30 	& 0.20 	& 50\\
		\texttt{iter-30 						}	& 100 & 30 	& 0.20 	& 30\\
		\texttt{iter-30+elite-10 				}	& 100 & 30 	& 0.10 	& 30\\
		\texttt{iter-50+elite-10 				}	& 100 & 30 	& 0.10 	& 50\\
		\texttt{iter-200+pop\_size-1000+elite-30} 	& 100 & 1000 	& 0.30 	& 200\\	
		\bottomrule
	\end{tabular}
	\captionof{table}{CMA-ES parameters}
	\label{tab:cmaes-param}
\end{figure}


Try different population size and learning rate and see what best performance you can obtain.

Try different number of generations. What is the minimum number of gener- ations that you can obtain a solution close enough to the global optimum

For each test function, plot the best and the worse fitness for each generation

\begin{figure}[htb]
	\centering
	
	\begin{tabular}{lccc}
		\toprule
		\textbf{experiment} & \textbf{best fitness} & \textbf{worse fitness} & \textbf{avg run time} \\
		\midrule
		\texttt{baseline 						}	&  677.09 &                677.09 &                 0.57 sec \\
		\texttt{pop\_size-100 					}	&   618.11 &                618.11 &                 0.65 sec \\
		\texttt{pop\_size-1000 					}	&    78.54 &                 78.56 &                 1.21 sec \\
		\texttt{elite-30 						}	&  690.86 &                690.86 &                 0.57 sec \\
		\texttt{elite-10 						}	&  712.47 &                712.47 &                 0.61 sec \\
		\texttt{pop\_size-100+elite-30 			}	&   611.05 &                611.06 &                 0.62 sec \\
		\texttt{pop\_size-100+elite-10 			}	&   672.11 &                672.11 &                 0.62 sec \\
		\texttt{pop\_size-1000+elite-30 		}	&    65.15 &                 65.25 &                 1.17 sec \\
		\texttt{pop\_size-1000+elite-10 		}	&   150.28 &                150.28 &                 1.08 sec \\
		\texttt{iter-50 						}	&  764.01 &                764.01 &                 0.43 sec \\
		\texttt{iter-30 						}	&   768.3 &                 768.3 &                 0.39 sec \\
		\texttt{iter-30+elite-10 				}	&  882.68 &                882.68 &                 0.39 sec \\
		\texttt{iter-50+elite-10 				}	&  729.77 &                729.77 &                 0.42 sec \\
		\texttt{iter-200+pop\_size-1000+elite-30} 	&    88.76 &                 88.77 &                 1.78 sec \\\bottomrule
	\end{tabular}
	\captionof{table}{Sphere CME-AS performance}
	\label{tab:cmeas-performance-s}
\end{figure}

\begin{figure}[htb]
	\centering
	
	\begin{tabular}{lccc}
		\toprule
		\textbf{experiment} & \textbf{best fitness} & \textbf{worse fitness} & \textbf{avg run time} \\
		\midrule
			\texttt{baseline 						}	&    1493.64 &               1493.64 &                 0.75 sec \\
		\texttt{pop\_size-100 					}	&      1404.3 &                1404.3 &                 0.84 sec \\
		\texttt{pop\_size-1000 					}	&      528.97 &                530.01 &                 1.59 sec \\
		\texttt{elite-30 						}	&    1534.59 &               1534.59 &                 0.76 sec \\
		\texttt{elite-10 						}	&    1641.93 &               1641.93 &                 0.93 sec \\
		\texttt{pop\_size-100+elite-30 			}	&     1353.75 &               1353.77 &                 0.85 sec \\
		\texttt{pop\_size-100+elite-10 			}	&     1464.23 &               1464.23 &                  0.8 sec \\
		\texttt{pop\_size-1000+elite-30 		}	&      518.71 &                520.95 &                 1.57 sec \\
		\texttt{pop\_size-1000+elite-10 		}	&      719.32 &                719.49 &                 1.46 sec \\
		\texttt{iter-50 						}	&    1584.02 &               1584.07 &                 0.63 sec \\
		\texttt{iter-30 						}	&    1569.96 &               1570.12 &                 0.59 sec \\
		\texttt{iter-30+elite-10 				}	&    1641.79 &               1641.79 &                 0.59 sec \\
		\texttt{iter-50+elite-10 				}	&    1627.19 &               1627.19 &                 0.63 sec \\
		\texttt{iter-200+pop\_size-1000+elite-30} 	&      477.72 &                 477.8 &                 2.31 sec \\
		\bottomrule
	\end{tabular}
	\captionof{table}{Rastrigin CME-AS performance}
	\label{tab:cmeas-performance-r}
\end{figure}

		
			
\section{Benchmarking}
The comparison has been carried out using the following parameters for all algorithms: 
{domain size dimension} of 100, 5000 {population size}, 2000 {number of generations}, {elite set ratio of 0.20 and  {learning rate} of 1e-3.
\bigskip 

In Figures \ref{fig:best-s} and \ref{fig:best-r} are plotted the comparisons of CEM, NES and CMA-ES for the best fitness. 
\bigskip 


Regarding the sphere function, both CEM and CMA-ES perform well, while NES is much slower to converge. 
Ultimately, CMA-ES is better than CEM for the sphere function with these parameters.

\begin{figure}[H]
	\centering
	\includegraphics[width=.6\linewidth]{../src/out/comparison/sphere/best-comparison.pdf}	
	\caption{Comparison of sphere function with algorithm CEM, NES and CMA-ES for the best fitness}
	\label{fig:best-s}
\end{figure}

Regarding the rastrigin function, NES is definitely the worst algorithm. It is not able to converge to the global optimum.
Both CEM and CMA-ES perform well.
Initially, CEM goes down faster to the minimum, although it is not able to converge to 0 with these parameters, remaining only very close to the optimum. 
With about 200 iterations, CMA-ES converges to the global optimum, so it is certainly the best algorithm for this test function.

\bigskip

In Figures \ref{fig:best-s} and \ref{fig:best-r} are plotted the comparisons of CEM, NES and CMA-ES for the worst fitness. 

\begin{figure}[H]
	\centering
	\includegraphics[width=.6\linewidth]{../src/out/comparison/rastrigin/best-comparison.pdf}	
	\caption{Comparison of CEM, NES and CMA-ES for the best rastrigin fitness}
	\label{fig:best-r}
\end{figure}

Regarding the sphere function, the algorithms behave as in the case of best fitness. Both CEM and CMA-ES perform well, while NES is much slower to converge. 

\begin{figure}[H]
	\centering
	\includegraphics[width=.6\linewidth]{../src/out/comparison/sphere/worst-comparison.pdf}	
	\caption{Comparison of sphere function with algorithm CEM, NES and CMA-ES for the worse fitness}
	\label{fig:worst-s}
\end{figure}


Also for the rastrigin function, the algorithms behave as in the case of best fitness. NES is definitely the worst algorithm. It is not able to converge to the global optimum.
Both CEM and CMA-ES perform well, although CMA-ES converges to the global optimum in about 150 generations.

\begin{figure}[H]
	\centering
	\includegraphics[width=.6\linewidth]{../src/out/comparison/rastrigin/worst-comparison.pdf}	
	\caption{Comparison of CEM, NES and CMA-ES for the worse rastrigin fitness}
	\label{fig:worst-r}
\end{figure}

\end{document}
