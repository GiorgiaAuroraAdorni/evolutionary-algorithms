\documentclass[a4paper,12pt]{article} % This defines the style of your paper

\usepackage[top = 2.5cm, bottom = 2.5cm, left = 2.5cm, right = 2.5cm]{geometry} 
\usepackage[utf8]{inputenc} %utf8 % lettere accentate da tastiera
\usepackage[english]{babel} % lingua del documento
\usepackage[T1]{fontenc} % codifica dei font

\usepackage{multirow} % Multirow is for tables with multiple rows within one 
%cell.
\usepackage{booktabs} % For even nicer tables.

\usepackage{graphicx} 

\usepackage{setspace}
\setlength{\parindent}{0in}

\usepackage{float}

\usepackage{fancyhdr}

\usepackage{caption}
\usepackage{amssymb}
\usepackage{amsmath}
\usepackage{mathtools}
\usepackage{color}

\usepackage[hidelinks]{hyperref}
\usepackage{csquotes}
\usepackage{subfigure}

\usepackage{ifxetex,ifluatex}
\usepackage{etoolbox}
\usepackage[svgnames]{xcolor}

\usepackage{tikz}

\usepackage{framed}

 \newcommand*\quotefont{\fontfamily{LinuxLibertineT-LF}} % selects Libertine as 
 %the quote font


\newcommand*\quotesize{40} % if quote size changes, need a way to make shifts 
%relative
% Make commands for the quotes
\newcommand*{\openquote}
{\tikz[remember picture,overlay,xshift=-4ex,yshift=-1ex]
	\node (OQ) 
	{\quotefont\fontsize{\quotesize}{\quotesize}\selectfont``};\kern0pt}

\newcommand*{\closequote}[1]
{\tikz[remember picture,overlay,xshift=4ex,yshift=-1ex]
	\node (CQ) {\quotefont\fontsize{\quotesize}{\quotesize}\selectfont''};}

% select a colour for the shading
\colorlet{shadecolor}{WhiteSmoke}

\newcommand*\shadedauthorformat{\emph} % define format for the author argument

% Now a command to allow left, right and centre alignment of the author
\newcommand*\authoralign[1]{%
	\if#1l
	\def\authorfill{}\def\quotefill{\hfill}
	\else
	\if#1r
	\def\authorfill{\hfill}\def\quotefill{}
	\else
	\if#1c
	\gdef\authorfill{\hfill}\def\quotefill{\hfill}
	\else\typeout{Invalid option}
	\fi
	\fi
	\fi}
% wrap everything in its own environment which takes one argument (author) and 
%one optional argument
% specifying the alignment [l, r or c]
%
\newenvironment{shadequote}[2][l]%
{\authoralign{#1}
	\ifblank{#2}
	{\def\shadequoteauthor{}\def\yshift{-2ex}\def\quotefill{\hfill}}
	{\def\shadequoteauthor{\par\authorfill\shadedauthorformat{#2}}\def\yshift{2ex}}
	\begin{snugshade}\begin{quote}\openquote}
		{\shadequoteauthor\quotefill\closequote{\yshift}\end{quote}\end{snugshade}}

\newcommand{\footref}[1]{%
	$^{\ref{#1}}$%
}

\pagestyle{fancy}

\setlength\parindent{24pt}

\fancyhf{}

\lhead{\footnotesize Machine Learning: Assignment 5}

\rhead{\footnotesize Giorgia Adorni}

\cfoot{\footnotesize \thepage} 

\begin{document}
%	\thispagestyle{empty}  
	\noindent{
	\begin{tabular}{p{15cm}} 
		{\large \bf Machine Learning} \\
		Università della Svizzera Italiana \\ Faculty of Informatics \\ \today  \\
		\hline
		\\
	\end{tabular} 
	
	\vspace*{0.3cm} 
	
	\begin{center}
		{\Large \bf Assignment 5: Evolutionary Algorithms}
		\vspace{2mm}
		
		{\bf Giorgia Adorni (giorgia.adorni@usi.ch)}
		
	\end{center}  
}
	\vspace{0.4cm}

	%%%%%%%%%%%%%%%%%%%%%%%%%%%%%%%%%%%%%%%%%%%%%%%%
	%%%%%%%%%%%%%%%%%%%%%%%%%%%%%%%%%%%%%%%%%%%%%%%%

\section{Introduction}
In this assignment, we are going to implement a few evolutionary algorithms against test functions and analyze the characteristics of different algorithms.
Let $f : \mathbb{R}^n \rightarrow \mathbb{R}$ be a test function with n-dimensional domain.

\noindent\textbf{Sphere function}:
	\begin{equation}
		f(\overline{x}) = \sum_{i=1}^nx_i^2 \mbox{.}
	\end{equation}
\textbf{Rastrigin functionction}:
	\begin{equation}
	f(\overline{x}) = An + \sum_{i=1}^n [x_i^2-A \cos(2\pi x_i)]
	\end{equation}
	where $A = 10$. The search domain is constraint as $x_i \in [-5, 5]$.

\section{Test Functions}

In Figure \ref{fig:s} and \ref{fig:r} are shown the 2D contour-plot of 2-dimensional test functions.

\begin{figure}[htb]
	\begin{minipage}[b]{.49\textwidth}
		\centering
		\includegraphics[width=\linewidth]{../src/out/sphere_test.pdf}	
		\caption{2D contour plot of 2D Sphere}
		\label{fig:s}
	\end{minipage}
	~
	\begin{minipage}[b]{.49\textwidth}
		\centering
		\includegraphics[width=\linewidth]{../src/out/rastrigin_test.pdf}	
		\caption{2D contour plot of 2D Rastrigin}
		\label{fig:r}
	\end{minipage}
\end{figure}

\begin{figure}[htb]
	\begin{minipage}[b]{.49\textwidth}
		\centering
		\includegraphics[width=\linewidth]{../src/out/sphere_test-eval.pdf}	
		\caption{Sphere test function evaluation}
		\label{fig:s-eval}
	\end{minipage}
	~
	\begin{minipage}[b]{.49\textwidth}
		\centering
		\includegraphics[width=\linewidth]{../src/out/rastrigin_test-eval.pdf}	
		\caption{Rastrigin test function evaluation}
		\label{fig:r-eval}
	\end{minipage}
\end{figure}

For each test function, 100 points has been uniformly sampled in the domain and evaluated with the test function. This points are shown in Figure \ref{fig:s-eval} and \ref{fig:r-eval}. 
Filling the contours and showing the colour-map it is easier to see which are the regions of the global optimum. All the points with value close to 0, the ones in purple, belong to the global optimal region.


For the sphere, the points are concentrated in the central region of the figure. 

For the rastrigin function evaluate the samples is many more complicated, since the function has many local minima. However, the global optimum is located in the centre of plot. For this reason, it will be seen later that finding the global optimum of this function will be difficult.
 
\section{Cross-Entropy Method (CEM)}
\begin{figure}[H]
	\centering
	
	\begin{tabular}{lccc}
		\toprule
		\textbf{experiment} & \textbf{population} & \textbf{elite} &
		\textbf{generations} \\
		\midrule
		\texttt{baseline 						}	 & 100 	& 0.20 	& 100\\
		\texttt{pop\_size-1000 					}	 & 1000 	& 0.20 	& 100\\
		\texttt{pop\_size-3000 					}	 & 3000 	& 0.20 	& 100\\
		\texttt{elite-40 						}	 & 100 	& 0.40 	& 100\\
		\texttt{elite-30 						}	 & 100 	& 0.30 	& 100\\
		\texttt{elite-10 						}	 & 100 	& 0.10 	& 100\\
		\texttt{pop\_size-1000+elite-40 		}	 & 1000 	& 0.40 	& 100\\
		\texttt{pop\_size-1000+elite-30 		}	 & 1000 	& 0.30 	& 100\\
		\texttt{pop\_size-1000+elite-10 		}	 & 1000 	& 0.10 	& 100\\
		\texttt{pop\_size-3000+elite-40 		}	 & 3000 	& 0.40 	& 100\\
		\texttt{pop\_size-3000+elite-30 		}	 & 3000 	& 0.30 	& 100\\
		\texttt{pop\_size-3000+elite-10 		}	 & 3000 	& 0.10 	& 100\\
		\texttt{iter-200 						}	 & 100 	& 0.20 	& 200\\
		\texttt{iter-50 						}	 & 100 	& 0.20 	& 50\\
		\texttt{iter-30 						}	 & 100 	& 0.20 	& 30\\
		\texttt{iter-200+elite-30 				}	 & 100 	& 0.30 	& 200\\
		\texttt{iter-200+pop\_size-3000+elite-30} 	 & 3000 	& 0.30 	& 200\\	
		\bottomrule
	\end{tabular}
	\captionof{table}{CEM parameters}
	\label{tab:cem-param}
\end{figure}

The first algorithm implemented is the \textbf{Cross-Entropy Method (CEM)}. 

Different execution of the algorithm has been performed. The first experiment is called baseline, and it uses the same parameters as for the following algorithms. After that, were made some attempts of improving the performances trying different population size and elite set ratio, and also with different number of generations. In all the experiments the domain size is fixed to 100 dimensions.

For all the experiments, the initial population parameters are initialised reasonably far from the global optimum, in particular, the mean is uniformly sampled in the range [-5, 5] and the variance is uniformly sampled in the range [4, 5].

In Table \ref{tab:cem-param} are shown the different combinations of parameters used.
\bigskip

The algorithm has been run 3 times for both test functions. 
In order to evaluate the performance, for each pair of experiment and test function, the best and the worse fitness for each generation (averaged over 3 runs) has been plotted. 
 
In Tables \ref{tab:cem-performance-s} and \ref{tab:cem-performance-r} are summarised the results. 
\bigskip

\begin{figure}[htb]
	\centering
	
	\begin{tabular}{lccc}
		\toprule
		\textbf{experiment} & \textbf{best fitness} & \textbf{worse fitness} & \textbf{avg run time} \\
		\midrule
		\texttt{baseline 						}		 &	  7.17 &	   7.19 &	  0.43 sec \\
		\texttt{pop\_size-1000 					}		&	   0.0 &	    0.0 &	  0.94 sec \\
		\texttt{pop\_size-3000 					}		&	   0.0 &	    0.0 &	  2.24 sec \\
		\texttt{elite-40 						}		 &	  2.65 &	   2.71 &	  0.54 sec \\
		\texttt{elite-30 						}		 &	  1.54 &	   1.56 &	  0.42 sec \\
		\texttt{elite-10 						}		 &	 93.74 &	  93.74 &	  0.42 sec \\
		\texttt{pop\_size-1000+elite-40 		}	 &	   0.0 &	    0.0 &	  0.95 sec \\
		\texttt{pop\_size-1000+elite-30 		}	 &	   0.0 &	    0.0 &	  0.98 sec \\
		\texttt{pop\_size-1000+elite-10 		}	 &	   0.0 &	    0.0 &	  0.95 sec \\
		\texttt{pop\_size-3000+elite-40 		}	 &	   0.0 &	    0.0 &	  2.17 sec \\
		\texttt{pop\_size-3000+elite-30 		}	 &	   0.0 &	    0.0 &	  2.27 sec \\
		\texttt{pop\_size-3000+elite-10 		}	 &	   0.0 &	    0.0 &	  2.24 sec \\
		\texttt{iter-200 						}		&	  8.57 &	   8.57 &	  0.62 sec \\
		\texttt{iter-50 						}	&	 12.09 &	   12.9 &	  0.43 sec \\
		\texttt{iter-30 						}	&	 27.18 &	  35.71 &	  0.36 sec \\
		\texttt{iter-200+elite-30 				}	&	  0.18 &	   0.18 &	   0.5 sec \\
		\texttt{iter-200+pop\_size-3000+elite-30} 	 &	   0.0 &	    0.0 &	  3.63 sec \\
		\bottomrule
	\end{tabular}
	\captionof{table}{Sphere CEM performance}
	\label{tab:cem-performance-s}
\end{figure}

In Figure \ref{fig:cem-s-fitness/baseline} is plotted the best and the worse fitness for each generation (averaged over 3 runs) for the baseline model. Below, in Figures \ref{fig:cem-s-fitness/1000} and \ref{fig:cem-s-fitness/200}, are plotted the fitness of the two models that perform better with the sphere function.

For the sphere function, it is possible to see that simply using a large enough population size the algorithm converges. 
Moreover, increasing the elite set ratio the algorithm converges quickly.
 
 \begin{figure}[H]
 	\centering
 	\begin{minipage}[b]{.6\textwidth}
 		\includegraphics[width=\linewidth]{../src/out/cem/baseline/sphere/fitness.pdf}	
 	\end{minipage}
	 \caption{Sphere fitness \texttt{baseline}}
	 \label{fig:cem-s-fitness/baseline}
	 
	 \begin{minipage}[b]{.6\textwidth}
	 	\includegraphics[width=\linewidth]{../src/out/cem/pop_size-1000/sphere/fitness.pdf}	
	 \end{minipage}
	 \caption{Sphere fitness \texttt{pop\_size-1000}}
	 \label{fig:cem-s-fitness/1000}
	 
	 \begin{minipage}[b]{.6\textwidth}
	 	\includegraphics[width=\linewidth]{../src/out/cem/iter-200+elite-30/sphere/fitness.pdf}	
	 \end{minipage}
	 \caption{Sphere fitness \texttt{iter-200+elite-30}}
	 \label{fig:cem-s-fitness/200}
\end{figure}

In the following Figures is shown the effect of executing the algorithm on the experiment \texttt{pop\_size-1000}. 
 \begin{figure}[H]
	\begin{minipage}[b]{.3\textwidth}
		\centering
		\includegraphics[width=\linewidth]{../src/out/cem/pop_size-1000/sphere/cem-generation-contour-0.pdf}	
	\end{minipage}
	~
	\begin{minipage}[b]{.3\textwidth}
		\centering
		\includegraphics[width=\linewidth]{../src/out/cem/pop_size-1000/sphere/cem-generation-contour-10.pdf}	
	\end{minipage}
	~
	\begin{minipage}[b]{.3\textwidth}
		\centering
		\includegraphics[width=\linewidth]{../src/out/cem/pop_size-1000/sphere/cem-generation-contour-20.pdf}	
	\end{minipage}

	\begin{minipage}[b]{.3\textwidth}
		\centering
		\includegraphics[width=\linewidth]{../src/out/cem/pop_size-1000/sphere/cem-generation-contour-30.pdf}	
	\end{minipage}
	~
	\begin{minipage}[b]{.3\textwidth}
		\centering
		\includegraphics[width=\linewidth]{../src/out/cem/pop_size-1000/sphere/cem-generation-contour-40.pdf}	
	\end{minipage}
	~
	\begin{minipage}[b]{.3\textwidth}
		\centering
		\includegraphics[width=\linewidth]{../src/out/cem/pop_size-1000/sphere/cem-generation-contour-50.pdf}	
	\end{minipage}
	
		\begin{minipage}[b]{.3\textwidth}
		\centering
		\includegraphics[width=\linewidth]{../src/out/cem/pop_size-1000/sphere/cem-generation-contour-60.pdf}	
	\end{minipage}
	~
	\begin{minipage}[b]{.3\textwidth}
		\centering
		\includegraphics[width=\linewidth]{../src/out/cem/pop_size-1000/sphere/cem-generation-contour-70.pdf}	
	\end{minipage}
	~
	\begin{minipage}[b]{.3\textwidth}
		\centering
		\includegraphics[width=\linewidth]{../src/out/cem/pop_size-1000/sphere/cem-generation-contour-80.pdf}	
	\end{minipage}
	
\end{figure}

For the rastrigin function, it is necessary not only to increase the size of the population but also the number of generations to allow the algorithm to converge. Moreover, in this case, by decreasing the elite set ratio the algorithm performs better.  

\begin{figure}[htb]
	\centering
	
	\begin{tabular}{lccc}
		\toprule
		\textbf{experiment} & \textbf{best fitness} & \textbf{worse fitness} & \textbf{avg run time} \\
		\midrule
		\texttt{baseline 						}		 &	 146.9 &	 147.84 &	  0.74 sec \\
		\texttt{pop\_size-1000 					}			   &	408.33 &	  726.5 &	  1.39 sec \\
		\texttt{pop\_size-3000 					}			   &	561.48 &	  998.0 &	   3.0 sec \\
		\texttt{elite-40 						}		 &	183.21 &	 308.04 &	  0.64 sec \\
		\texttt{elite-30 						}		 &	143.56 &	 173.52 &	  0.68 sec \\
		\texttt{elite-10 						}		 &	342.08 &	 342.11 &	  0.66 sec \\
		\texttt{pop\_size-1000+elite-40 		}	  &	 754.4 &	1225.99 &	  1.37 sec \\
		\texttt{pop\_size-1000+elite-30 		}	  &	647.67 &	1084.44 &	  1.37 sec \\
		\texttt{pop\_size-1000+elite-10 		}	  &	 70.65 &	 194.74 &	   1.4 sec \\
		\texttt{pop\_size-3000+elite-40 		}	  &	774.94 &	1290.65 &	  3.08 sec \\
		\texttt{pop\_size-3000+elite-30 		}	  &	740.86 &	1245.91 &	  3.37 sec \\
		\texttt{pop\_size-3000+elite-10 		}	  &	117.32 &	 317.21 &	   2.9 sec \\
		\texttt{iter-200 						}	 &	134.35 &	 134.36 &	  0.88 sec \\
		\texttt{iter-50 						}	  &	449.25 &	 641.98 &	   0.8 sec \\
		\texttt{iter-30 						}	  &	831.28 &	1178.28 &	  0.64 sec \\
		\texttt{iter-200+elite-30 				}	 &	 83.65 &	  83.76 &	  0.74 sec \\
		\texttt{iter-200+pop\_size-3000+elite-30} 	&	 15.85 &	   45.2 &	  4.96 sec \\
		\bottomrule
	\end{tabular}
	\captionof{table}{Rastrigin CEM performance}
	\label{tab:cem-performance-r}
\end{figure}
\bigskip

In Figure \ref{fig:cem-r-fitness/baseline} is plotted the best and the worse fitness for each generation (averaged over 3 runs) for the baseline model. Below, in Figures \ref{fig:cem-r-fitness/1000} and \ref{fig:cem-r-fitness/200}, are plotted the fitness of the two models that perform better with the rastrigin function.

\begin{figure}[H]
	\centering
	\begin{minipage}[b]{.6\textwidth}
		\includegraphics[width=\linewidth]{../src/out/cem/baseline/rastrigin/fitness.pdf}	
	\end{minipage}
	\caption{Sphere fitness \texttt{baseline}}
	\label{fig:cem-r-fitness/baseline}
\end{figure}
\begin{figure}[htb]
	\centering
	\begin{minipage}[b]{.6\textwidth}
		\includegraphics[width=\linewidth]{../src/out/cem/pop_size-1000+elite-10/rastrigin/fitness.pdf}	
	\end{minipage}
	\caption{Sphere fitness \texttt{pop\_size-1000}}
	\label{fig:cem-r-fitness/1000}
\end{figure}
\begin{figure}[htb]
	\centering	
	\begin{minipage}[b]{.6\textwidth}
		\includegraphics[width=\linewidth]{../src/out/cem/pop_size-3000+iter-200+elite-30/rastrigin/fitness.pdf}	
	\end{minipage}
	\caption{Sphere fitness \texttt{iter-200+elite-30}}
	\label{fig:cem-r-fitness/200}
\end{figure}

\bigskip
For the sphere function, 40 generations are enough to obtain a solution close enough to the global optimum, as shown for the experiment \texttt{pop\_size-1000}.

For the rastrigin function and the parameter tested, at least 200 generations are necessary to obtain a solution close enough to the global optimum, as shown for the experiment \texttt{iter-200+elite-30}.
\section{Natural Evolution Strategy (NES)}

\begin{figure}[htb]
	\centering
	
	\begin{tabular}{lcccc}
		\toprule
		\textbf{experiment} & \textbf{domain} & \textbf{popul.} & \textbf{learning rate} &
		\textbf{generations} \\
		\midrule
		\texttt{baseline 						}	& 100 & 30 		& 1e-2 	& 100\\
		\texttt{pop\_size-100 					}	& 100 & 100 	& 1e-2 	& 100\\
		\texttt{pop\_size-1000 					}	& 100 & 1000 	& 1e-2 	& 100\\
		\texttt{pop\_size-5000 					}	& 100 & 5000 	& 1e-2 	& 100\\
		\texttt{lr-001 							}	& 100 & 30 		& 1e-3 	& 100\\
		\texttt{lr-0001	 						}	& 100 & 30 		& 1e-4 	& 100\\
		\texttt{lr-00001	 					}	& 100 & 30 		& 1e-5 	& 100\\
		\texttt{pop\_size-1000+lr-001 			}	& 100 & 1000 	& 1e-3 	& 100\\
		\texttt{pop\_size-1000+lr-0001 			}	& 100 & 1000 	& 1e-4 	& 100\\
		\texttt{pop\_size-5000+lr-001 			}	& 100 & 5000 	& 1e-3 	& 100\\
		\texttt{pop\_size-5000+lr-0001 			}	& 100 & 5000 	& 1e-4 	& 100\\
		\texttt{iter-2000 						}	& 100 & 30 		& 1e-2 	& 2000\\
		\texttt{iter-5000 						}	& 100 & 30 		& 1e-2 	& 5000\\
		\texttt{iter-2000+pop-1000 	}	& 100 & 1000 	& 1e-2 	& 2000\\
		\texttt{iter-5000+pop-1000 	}	& 100 & 1000 	& 1e-2 	& 5000\\
		\texttt{iter-2000+pop-5000 	}	& 100 & 5000 	& 1e-2 	& 2000\\
		\texttt{iter-5000+pop-5000 	}	& 100 & 5000 	& 1e-2 	& 5000\\
		\texttt{iter-2000+pop-1000+lr-001 }	& 100 & 1000 	& 1e-3 	& 2000\\
		\texttt{iter-2000+pop-1000+lr-0001 }	& 100 & 1000 	& 1e-4 	& 2000\\
		\texttt{iter-2000+pop-5000+lr-001 }	& 100 & 1000 	& 1e-3 	& 5000\\
		\texttt{iter-2000+pop-5000+lr-0001 }	& 100 & 1000 	& 1e-4 	& 5000\\
	
		\bottomrule
	\end{tabular}
	\captionof{table}{NES parameters}
	\label{tab:nes-param}
\end{figure}


Try different population size and learning rate and see what best performance you can obtain


Try different number of generations. What is the minimum number of gener- ations that you can obtain a solution close enough to the global optimum?


For each test function, plot the best and the worse fitness for each generation (averaged over 3 runs).


\begin{figure}[htb]
	\centering
	
	\begin{tabular}{lcccc}
		\toprule
		\textbf{experiment} & \textbf{domain} & \textbf{popul.} & \textbf{learning rate} &
		\textbf{generations} \\
		\midrule
		\texttt{baseline 						}	& 100 & 30 		& 1e-2 	& 100\\
		\texttt{pop\_size-100 					}	& 100 & 100 	& 1e-2 	& 100\\
		\texttt{pop\_size-1000 					}	& 100 & 1000 	& 1e-2 	& 100\\
		\texttt{pop\_size-5000 					}	& 100 & 5000 	& 1e-2 	& 100\\
		\texttt{lr-001 							}	& 100 & 30 		& 1e-3 	& 100\\
		\texttt{lr-0001	 						}	& 100 & 30 		& 1e-4 	& 100\\
		\texttt{lr-00001	 					}	& 100 & 30 		& 1e-5 	& 100\\
		\texttt{pop\_size-1000+lr-001 			}	& 100 & 1000 	& 1e-3 	& 100\\
		\texttt{pop\_size-1000+lr-0001 			}	& 100 & 1000 	& 1e-4 	& 100\\
		\texttt{pop\_size-5000+lr-001 			}	& 100 & 5000 	& 1e-3 	& 100\\
		\texttt{pop\_size-5000+lr-0001 			}	& 100 & 5000 	& 1e-4 	& 100\\
		\texttt{iter-2000 						}	& 100 & 30 		& 1e-2 	& 2000\\
		\texttt{iter-5000 						}	& 100 & 30 		& 1e-2 	& 5000\\
		\texttt{iter-2000+pop-1000 	}	& 100 & 1000 	& 1e-2 	& 2000\\
		\texttt{iter-5000+pop-1000 	}	& 100 & 1000 	& 1e-2 	& 5000\\
		\texttt{iter-2000+pop-5000 	}	& 100 & 5000 	& 1e-2 	& 2000\\
		\texttt{iter-5000+pop-5000 	}	& 100 & 5000 	& 1e-2 	& 5000\\
		\texttt{iter-2000+pop-1000+lr-001 }	& 100 & 1000 	& 1e-3 	& 2000\\
		\texttt{iter-2000+pop-1000+lr-0001 }	& 100 & 1000 	& 1e-4 	& 2000\\
		\texttt{iter-2000+pop-5000+lr-001 }	& 100 & 1000 	& 1e-3 	& 5000\\
		\texttt{iter-2000+pop-5000+lr-0001 }	& 100 & 1000 	& 1e-4 	& 5000\\
		\bottomrule
	\end{tabular}
	\captionof{table}{Sphere NES performance}
	\label{tab:nes-performance-s}
\end{figure}


\begin{figure}[htb]
	\centering
	
	\begin{tabular}{lcccc}
		\toprule
		\textbf{experiment} & \textbf{domain} & \textbf{popul.} & \textbf{learning rate} &
		\textbf{generations} \\
		\midrule
		\texttt{baseline 						}	& 100 & 30 		& 1e-2 	& 100\\
		\texttt{pop\_size-100 					}	& 100 & 100 	& 1e-2 	& 100\\
		\texttt{pop\_size-1000 					}	& 100 & 1000 	& 1e-2 	& 100\\
		\texttt{pop\_size-5000 					}	& 100 & 5000 	& 1e-2 	& 100\\
		\texttt{lr-001 							}	& 100 & 30 		& 1e-3 	& 100\\
		\texttt{lr-0001	 						}	& 100 & 30 		& 1e-4 	& 100\\
		\texttt{lr-00001	 					}	& 100 & 30 		& 1e-5 	& 100\\
		\texttt{pop\_size-1000+lr-001 			}	& 100 & 1000 	& 1e-3 	& 100\\
		\texttt{pop\_size-1000+lr-0001 			}	& 100 & 1000 	& 1e-4 	& 100\\
		\texttt{pop\_size-5000+lr-001 			}	& 100 & 5000 	& 1e-3 	& 100\\
		\texttt{pop\_size-5000+lr-0001 			}	& 100 & 5000 	& 1e-4 	& 100\\
		\texttt{iter-2000 						}	& 100 & 30 		& 1e-2 	& 2000\\
		\texttt{iter-5000 						}	& 100 & 30 		& 1e-2 	& 5000\\
		\texttt{iter-2000+pop-1000 	}	& 100 & 1000 	& 1e-2 	& 2000\\
		\texttt{iter-5000+pop-1000 	}	& 100 & 1000 	& 1e-2 	& 5000\\
		\texttt{iter-2000+pop-5000 	}	& 100 & 5000 	& 1e-2 	& 2000\\
		\texttt{iter-5000+pop-5000 	}	& 100 & 5000 	& 1e-2 	& 5000\\
		\texttt{iter-2000+pop-1000+lr-001 }	& 100 & 1000 	& 1e-3 	& 2000\\
		\texttt{iter-2000+pop-1000+lr-0001 }	& 100 & 1000 	& 1e-4 	& 2000\\
		\texttt{iter-2000+pop-5000+lr-001 }	& 100 & 1000 	& 1e-3 	& 5000\\
		\texttt{iter-2000+pop-5000+lr-0001 }	& 100 & 1000 	& 1e-4 	& 5000\\
		\bottomrule
	\end{tabular}
	\captionof{table}{Rastrigin NES performance}
	\label{tab:nes-performance-r}
\end{figure}

\section{Covariance Matrix Adaptation Evolution Strategy \\(CMA-ES)}

The last algorithm implemented is the \textbf{Covariance Matrix Adaptation Evolution Strategy (CMA-ES)}. 

Different execution of the algorithm has been performed using the same parameters employed with the first method. In Table \ref{tab:cmaes-param} are shown the different combinations of parameters used.

\begin{figure}[htb]
	\centering
	
	\begin{tabular}{lccc}
		\toprule
		\textbf{experiment} & \textbf{population} & \textbf{elite} &
		\textbf{generations} \\
		\midrule
		\texttt{baseline 						}	 & 100 	& 0.20 	& 100\\
		\texttt{pop\_size-1000 					}	 & 1000 	& 0.20 	& 100\\
		\texttt{pop\_size-3000 					}	 & 3000 	& 0.20 	& 100\\
		\texttt{elite-40 						}	 & 100 	& 0.40 	& 100\\
		\texttt{elite-30 						}	 & 100 	& 0.30 	& 100\\
		\texttt{elite-10 						}	 & 100 	& 0.10 	& 100\\
		\texttt{pop\_size-1000+elite-40 		}	 & 1000 	& 0.40 	& 100\\
		\texttt{pop\_size-1000+elite-30 		}	 & 1000 	& 0.30 	& 100\\
		\texttt{pop\_size-1000+elite-10 		}	 & 1000 	& 0.10 	& 100\\
		\texttt{pop\_size-3000+elite-40 		}	 & 3000 	& 0.40 	& 100\\
		\texttt{pop\_size-3000+elite-30 		}	 & 3000 	& 0.30 	& 100\\
		\texttt{pop\_size-3000+elite-10 		}	 & 3000 	& 0.10 	& 100\\
		\texttt{iter-200 						}	 & 100 	& 0.20 	& 200\\
		\texttt{iter-50 						}	 & 100 	& 0.20 	& 50\\
		\texttt{iter-30 						}	 & 100 	& 0.20 	& 30\\
		\texttt{iter-200+elite-30 				}	 & 100 	& 0.30 	& 200\\
		\texttt{iter-200+pop\_size-3000+elite-30} 	 & 3000 	& 0.30 	& 200\\	
		\bottomrule
	\end{tabular}
	\captionof{table}{CMA-ES parameters}
	\label{tab:cmaes-param}
\end{figure}

The algorithm has been run 3 times for both test functions. 
In order to evaluate the performance, for each pair of experiment and test function, the best and the worse fitness for each generation (averaged over 3 runs) has been plotted. 

In Tables \ref{tab:cmaes-performance-s} and \ref{tab:cmaes-performance-r} are summarised the results. 

\begin{figure}[htb]
	\centering
	
	\begin{tabular}{lccc}
		\toprule
		\textbf{experiment} & \textbf{best fitness} & \textbf{worse fitness} & \textbf{avg run time} \\
		\midrule
		\texttt{baseline 						}		&	567.81 &	 567.82 &	  0.63 sec \\
		\texttt{pop\_size-1000 					}		  &	 92.09 &	  92.11 &	  1.14 sec \\
		\texttt{pop\_size-3000 					}		  &	  0.01 &	   0.01 &	  2.56 sec \\
		\texttt{elite-40 						}		&	586.48 &	 586.48 &	  0.75 sec \\
		\texttt{elite-30 						}		&	570.19 &	 570.21 &	  0.86 sec \\
		\texttt{elite-10 						}		&	657.87 &	 657.87 &	   0.6 sec \\
		\texttt{pop\_size-1000+elite-40 		}	 &	 56.86 &	  57.09 &	   1.2 sec \\
		\texttt{pop\_size-1000+elite-30 		}	 &	  40.8 &	  40.84 &	   1.2 sec \\
		\texttt{pop\_size-1000+elite-10 		}	 &	132.82 &	 132.83 &	  1.09 sec \\
		\texttt{pop\_size-3000+elite-40 		}	 &	   0.0 &	   0.01 &	  2.58 sec \\
		\texttt{pop\_size-3000+elite-30 		}	 &	   0.0 &	    0.0 &	  2.43 sec \\
		\texttt{pop\_size-3000+elite-10 		}	 &	   6.4 &	   6.41 &	   2.3 sec \\
		\texttt{iter-200 						}		&	586.27 &	 586.27 &	  0.83 sec \\
		\texttt{iter-50 						}		 &	609.34 &	 609.35 &	  0.47 sec \\
		\texttt{iter-30 						}		 &	602.65 &	 602.69 &	  0.47 sec \\
		\texttt{iter-200+elite-30 				}	&	 554.4 &	  554.4 &	  0.84 sec \\
		\texttt{iter-200+pop\_size-3000+elite-30} 	 &	   0.0 &	    0.0 &	  4.87 sec \\
		
		\bottomrule
	\end{tabular}
	\captionof{table}{Sphere CMA-ES performance}
	\label{tab:cmaes-performance-s}
\end{figure}

For the sphere function, it is possible to see that simply using a large enough population size the algorithm converges.

In Figure \ref{fig:cmaes-s-fitness/baseline} is plotted the best and the worse fitness for each generation (averaged over 3 runs) for the baseline model. Below, in Figure \ref{fig:cmaes-s-fitness/3000} is plotted the fitness of the model that performs better with the sphere function.

 
\begin{figure}[H]
	\centering
	\begin{minipage}[b]{.6\textwidth}
		\includegraphics[width=\linewidth]{../src/out/cma_es/baseline/sphere/fitness.pdf}	
	\end{minipage}
	\caption{Sphere fitness \texttt{baseline}}
	\label{fig:cmaes-s-fitness/baseline}
\end{figure}
\begin{figure}[H]
	\centering
	\begin{minipage}[b]{.6\textwidth}
		\includegraphics[width=\linewidth]{../src/out/cma_es/pop_size-3000/sphere/fitness.pdf}	
	\end{minipage}
	\caption{Sphere fitness \texttt{pop\_size-3000}}
	\label{fig:cmaes-s-fitness/3000}
\end{figure}

For the rastrigin function, it is necessary not only to increase the size of the population but also the number of generations to allow the algorithm to converge.

\begin{figure}[htb]
	\centering
	
	\begin{tabular}{lccc}
		\toprule
		\textbf{experiment} & \textbf{best fitness} & \textbf{worse fitness} & \textbf{avg run time} \\
		\midrule
		\texttt{baseline 						}		&              1394.23 &	1394.23 &	  0.95 sec \\
		\texttt{pop\_size-1000 					}		&	446.51 &	 447.11 &	  1.53 sec \\
		\texttt{pop\_size-3000 					}		&	120.04 &	 126.67 &	  3.51 sec \\
		\texttt{elite-40 						}		&              1298.12 &	1298.27 &	  0.86 sec \\
		\texttt{elite-30 						}		&              1433.35 &	1433.38 &	  0.87 sec \\
		\texttt{elite-10 						}		&              1385.37 &	1385.37 &	  0.82 sec \\
		\texttt{pop\_size-1000+elite-40 		}	 &	443.78 &	 453.89 &	   1.6 sec \\
		\texttt{pop\_size-1000+elite-30 		}	 &	390.73 &	 393.75 &	  1.58 sec \\
		\texttt{pop\_size-1000+elite-10 		}	 &	550.56 &	 550.63 &	  1.51 sec \\
		\texttt{pop\_size-3000+elite-40 		}	 &	400.15 &	 751.57 &	  3.42 sec \\
		\texttt{pop\_size-3000+elite-30 		}	 &	161.26 &	 222.73 &	  3.17 sec \\
		\texttt{pop\_size-3000+elite-10 		}	 &	228.41 &	 229.67 &	  3.04 sec \\
		\texttt{iter-200 						}		&              1356.23 &	1356.23 &	  1.09 sec \\
		\texttt{iter-50 						}		 &              1290.16 &	1290.59 &	  0.69 sec \\
		\texttt{iter-30 						}		 &              1393.38 &	1394.41 &	  0.66 sec \\
		\texttt{iter-200+elite-30 				}	&              1229.94 &	1230.02 &	  1.11 sec \\
		\texttt{iter-200+pop\_size-3000+elite-30} 	 &	101.37 &	 103.35 &	  6.42 sec \\
		\bottomrule
	\end{tabular}
	\captionof{table}{Rastrigin CMA-ES performance}
	\label{tab:cmaes-performance-r}
\end{figure}


In Figure \ref{fig:cmaes-r-fitness/baseline} is plotted the best and the worse fitness for each generation (averaged over 3 runs) for the baseline model. Below, in Figures \ref{fig:cmaes-r-fitness/3000} and \ref{fig:cmaes-r-fitness/3000iter} are plotted the fitness of the models that perform better with the rastrigin function.

\begin{figure}[H]
	\centering
	\begin{minipage}[b]{.6\textwidth}
		\includegraphics[width=\linewidth]{../src/out/cma_es/baseline/rastrigin/fitness.pdf}	
	\end{minipage}
	\caption{Rastrigin fitness \texttt{baseline}}
	\label{fig:cmaes-r-fitness/baseline}

	\begin{minipage}[b]{.6\textwidth}
		\includegraphics[width=\linewidth]{../src/out/cma_es/pop_size-3000/rastrigin/fitness.pdf}	
	\end{minipage}
	\caption{Rastrigin fitness \texttt{pop\_size-3000}}
	\label{fig:cmaes-r-fitness/3000}
	
		\begin{minipage}[b]{.6\textwidth}
		\includegraphics[width=\linewidth]{../src/out/cma_es/pop_size-3000+iter-200+elite-30/rastrigin/fitness.pdf}	
	\end{minipage}
	\caption{Rastrigin fitness \texttt{pop\_size-3000+iter-200+elite-30}}
	\label{fig:cmaes-r-fitness/3000iter}
\end{figure}

In the following Figures is shown the effect of executing the algorithm on the experiment \texttt{pop\_size-3000+iter-200+elite-30}. 
\begin{figure}[H]
	\begin{minipage}[b]{.3\textwidth}
		\centering
		\includegraphics[width=\linewidth]{../src/out/cma_es/pop_size-3000+iter-200+elite-30/rastrigin/cma_es-generation-contour-0.pdf}	
	\end{minipage}
	~
	\begin{minipage}[b]{.3\textwidth}
		\centering
		\includegraphics[width=\linewidth]{../src/out/cma_es/pop_size-3000+iter-200+elite-30/rastrigin/cma_es-generation-contour-20.pdf}	
	\end{minipage}
	~
	\begin{minipage}[b]{.3\textwidth}
		\centering
		\includegraphics[width=\linewidth]{../src/out/cma_es/pop_size-3000+iter-200+elite-30/rastrigin/cma_es-generation-contour-40.pdf}	
	\end{minipage}
	
	\begin{minipage}[b]{.3\textwidth}
		\centering
		\includegraphics[width=\linewidth]{../src/out/cma_es/pop_size-3000+iter-200+elite-30/rastrigin/cma_es-generation-contour-60.pdf}	
	\end{minipage}
	~
	\begin{minipage}[b]{.3\textwidth}
		\centering
		\includegraphics[width=\linewidth]{../src/out/cma_es/pop_size-3000+iter-200+elite-30/rastrigin/cma_es-generation-contour-80.pdf}	
	\end{minipage}
	~
	\begin{minipage}[b]{.3\textwidth}
		\centering
		\includegraphics[width=\linewidth]{../src/out/cma_es/pop_size-3000+iter-200+elite-30/rastrigin/cma_es-generation-contour-100.pdf}	
	\end{minipage}
	
	\begin{minipage}[b]{.3\textwidth}
		\centering
		\includegraphics[width=\linewidth]{../src/out/cma_es/pop_size-3000+iter-200+elite-30/rastrigin/cma_es-generation-contour-120.pdf}	
	\end{minipage}
	~
	\begin{minipage}[b]{.3\textwidth}
		\centering
		\includegraphics[width=\linewidth]{../src/out/cma_es/pop_size-3000+iter-200+elite-30/rastrigin/cma_es-generation-contour-140.pdf}	
	\end{minipage}
	~
	\begin{minipage}[b]{.3\textwidth}
		\centering
		\includegraphics[width=\linewidth]{../src/out/cma_es/pop_size-3000+iter-200+elite-30/rastrigin/cma_es-generation-contour-160.pdf}	
	\end{minipage}
	
\end{figure}

For the sphere function, 40 generations are enough to obtain a solution close enough to the global optimum, as shown for the experiment \texttt{pop\_size-3000}.

For the rastrigin function and the parameter tested, it was not possible to converge to a global optimum. 
			
\section{Benchmarking}
The comparison has been carried out using the following parameters for all algorithms: 
{domain size dimension} of 100, 5000 {population size}, 2000 {number of generations}, {elite set ratio of 0.20 and  {learning rate} of 1e-3.
\bigskip 

In Figures \ref{fig:best-s} and \ref{fig:best-r} are plotted the comparisons of CEM, NES and CMA-ES for the best fitness. 
\bigskip 


Regarding the sphere function, both CEM and CMA-ES perform well, while NES is much slower to converge. 
Ultimately, CMA-ES is better than CEM for the sphere function with these parameters.

\begin{figure}[H]
	\centering
	\includegraphics[width=.6\linewidth]{../src/out/comparison/sphere/best-comparison.pdf}	
	\caption{Comparison of sphere function with algorithm CEM, NES and CMA-ES for the best fitness}
	\label{fig:best-s}
\end{figure}

Regarding the rastrigin function, NES is definitely the worst algorithm. It is not able to converge to the global optimum.
Both CEM and CMA-ES perform well.
Initially, CEM goes down faster to the minimum, although it is not able to converge to 0 with these parameters, remaining only very close to the optimum. 
With about 200 iterations, CMA-ES converges to the global optimum, so it is certainly the best algorithm for this test function.

\bigskip

In Figures \ref{fig:best-s} and \ref{fig:best-r} are plotted the comparisons of CEM, NES and CMA-ES for the worst fitness. 

\begin{figure}[H]
	\centering
	\includegraphics[width=.6\linewidth]{../src/out/comparison/rastrigin/best-comparison.pdf}	
	\caption{Comparison of CEM, NES and CMA-ES for the best rastrigin fitness}
	\label{fig:best-r}
\end{figure}

Regarding the sphere function, the algorithms behave as in the case of best fitness. Both CEM and CMA-ES perform well, while NES is much slower to converge. 

\begin{figure}[H]
	\centering
	\includegraphics[width=.6\linewidth]{../src/out/comparison/sphere/worst-comparison.pdf}	
	\caption{Comparison of sphere function with algorithm CEM, NES and CMA-ES for the worse fitness}
	\label{fig:worst-s}
\end{figure}


Also for the rastrigin function, the algorithms behave as in the case of best fitness. NES is definitely the worst algorithm. It is not able to converge to the global optimum.
Both CEM and CMA-ES perform well, although CMA-ES converges to the global optimum in about 150 generations.

\begin{figure}[H]
	\centering
	\includegraphics[width=.6\linewidth]{../src/out/comparison/rastrigin/worst-comparison.pdf}	
	\caption{Comparison of CEM, NES and CMA-ES for the worse rastrigin fitness}
	\label{fig:worst-r}
\end{figure}

\end{document}
